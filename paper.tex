% IMPORTANT: BECAUSE WE ARE USING MINTED, YOU MUST SAY pdflatex -shell-escape paper.tex

\documentclass[11pt, letterpaper, oneside, twocolumn] {article}
%\usepackage{fullpage}
\usepackage{minted}

\begin{document}

\title{HaPy \\
Haskell for Python
}
\author{David Fisher, Ashwin Siripurapu, and William Rowan}
\maketitle

\section{Introduction}

\section{Motivation}
It is a truth universally acknowledged, that a single Python module in possession of a good GUI, must be in want of a Haskell backend. Joking aside, three trends are clear when looking at the most recent State of Haskell survey and other data: in the first place, Haskellers primarily use Haskell for mathematical analysis, parsing/compiling, and (increasingly) web development; importantly, GUI development in Haskell is almost non-existent. Secondly, Python remains a popular language for creating GUIs, because of its high-level, human-readable syntax and intuitive ease of use. Lastly, Python code is much slower than the corresponding C code, while Haskell's performance is generally almost as good as C's for most use cases. This is in spite of the fact that Python and Haskell are both very high-level languages. 

\section{Use}

\section{Implementation}

\section{Further work}
In future, HaPy could support dynamic conversion from GADTs to Python types when instance of user-defined Haskell data are passed from the Haskell layer to the Python layer---in particular, Haskell's product types would map to Python classes while Haskell's sum types could be represented in Python by an enumerated type. 

\section{Refernces}
System.Plugins stuff.

\end{document}
